%===================================== CHAP 2 =================================

\chapter{Literature Review}
A generation ago the efficient market hypothesis was widely accepted by economists. E. F Fama (1970) emphasized the hypothesis that the capital markets are efficient, which means that the prices always fully reflect available information. Several studies in economic literature have investigated and concluded that the hypothesis is not true. Malkiel et al. (2003) argues that the market cannot be perfectly efficient, as pricing irregularities and predictable patterns in stock returns can appear and persist for short time periods. Grossman and Stiglitz (1980) argues that because information is costly, prices cannot perfectly reflect the available information, since if it did, investors would have not received compensation for the invested resources.
\\\\
Since market information is not directly available, Vlastakis et al. (2012) argues that studying how financial information affects the financial market requires some proxies for information flow. Several theories for information markets exist, but it turned out that it was hard to find proxies to measure before the internet and open data made it possible to find good proxies for financial information. Vlastakis et al. (2012)  separate financial information into information demand and supply. Measures of information supply try to capture the impact of news and announcements for investors, while measures of information demand try to capture the degree to which people are engaged in processing firm-related information.

\cleardoublepage