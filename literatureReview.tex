%===================================== CHAP 2 =================================

\chapter{Literature Review}
A generation ago, the efficient market hypothesis was widely accepted by economists. \cite{fama} emphasized the hypothesis that the capital markets are efficient, which means that the prices always fully reflect available information. Several studies in economic literature have investigated and concluded that this hypothesis is not true. \cite{malkiel} argues that the market cannot be perfectly efficient, as pricing irregularities and predictable patterns in stock returns can appear and persist for short time periods. \cite{grossman} argue that because information is costly, prices cannot perfectly reflect the available information, since if it did, investors would have not received compensation for the invested resources. Since information has a value, it is reasonable that there exist a market for buy and selling it. Several theories tries to explain how supply and demand in the information market is related to the financial market. 
\\\\
\cite{vlastakis} separate financial information flow into information demand and supply. Information supply tries to capture the amount of firm-related information being created or published. Information demand tries to capture the degree to which people are engaged in processing firm-related information. Since information demand and supply is not directly available, \cite{vlastakis} argue that studying how financial information affects the financial market requires proxies. Several theories for information markets exist, but until recently it has been hard to find good proxies for information flow. This has changed with the introduction of sources like Google Trends, online news databases, and Wikipedias Pageview Analysis.  
\\\\
Previous research has used several different measures for information demand. Among the most popular are Wikipedia views as used by \cite{moat} and different versions of Google Trends data. One of the most popular measures for information supply is daily or weekly news count, as used by \cite{vlastakis}. In the next paragraphs, we will present studies in economic literature about proxies for information demand and information supply and whether these can predict stock market developments. 
\\\\
Google Trends was introduced in 2008, and has been a popular proxy for information demand. Several studies in economic literature have investigated whether Google Trends data can predict stock market development. Most of the economic literature find that search volume and trading volume is correlated, but cannot find any significant correlation between search volume and volatility or stock return. \cite{preis} find a significant correlation between transaction volume and Google search volume of the corresponding company names, but it seems that neither buying transactions nor selling transactions are preferred when one detects an increased search volume. \cite{engelberg} find that an increase in Google search volume index (SVI) predicts higher stock returns in the next two weeks and an eventual price reversal within the year. \cite{challet2014} do not find any clear evidence that Google Trends data contains more predictability than price returns themselves. \cite{vlastakis} find that variations in information demand is positively correlated to trading volume and volatility. \cite{neri} investigate whether Google Trends data can predict stock market activity in Norway, and find that Google Trends data neither predict nor correlate with stock returns. 
\\\\
To collect Google Trends data, a keyword (or concept) is needed for each company. The two main keywords used in economic literature are stock sticker and company name, based on different arguments. \cite{joseph} use stock tickers. They argue that ticker trend is only used by someone who seriously is considering an investment, and that there are few other reasons for searching on company tickers. \cite{bijl} used search terms and argue that people searching for company information on the Internet are also those most likely to be interested in the company rather than an alternate meaning of the word. \cite{engelberg} use ticker trend, because ticker trends only capture investor attention on trading and asset pricing and that identifying a stock using its ticker also avoids the problem of multiple reference names. \cite{vlastakis} argue that company names avoid problems associated with the fact that several ticker names have generic meanings.For example, the stock ticker for Caterpillar is 'CAT'.
\\\\
Another common measure for information demand is Wikipedia views and edits for a company's Wikipedia page. \cite{moat} find that the Wikipedia view based trading strategies have significantly higher returns than the random trading strategies, while the Wikipedia edit based strategies do not have significantly higher returns than the random trading strategies. 
\\\\
News article count is a popular measure of information supply. \cite{preis2013} find a positive correlation between daily number of mentions of a company and the company's daily trading volume, both on the same day as the news is released and on the day before. \cite{ryan} find that company specific news have a significant impact on the corresponding company's price changes and trading volume. \cite{vlastakis} find a significant relationship between number of firm-specific news per week and realized volatility. \cite{tetlock} shows that public news predict lower ten-day reversals of daily stock returns and higher ten-day volume-induced momentum in daily returns. 
\\\\
Few studies have made an empirical comparison of the different information variables. Additionally, only a few studies have included more than one of the mentioned information variables in the analyses. \cite{vlastakis} and \cite{engelberg} are some of the few papers that have included more than type of information variables in their analyses. \cite{engelberg} have collected three types of keywords in Google Trends. These keywords are company name, the main product of the company and ticker trend. They have also included different types of transformed news counts in their models. \cite{engelberg} find that stock ticker has little correlation with a news-based measure of investor sentiment. \cite{engelberg} perform panel data regressions to check the predictive power of Google Trends data and news count with stock return as dependent variable. They get inconsistent results between the different sampling periods they perform panel data regressions. In the panel data regression with stock return as dependent variable for the sampling period 2004 to 2006, the independent variables for the ticker trend and main product are significant, while the transformed news story count is insignificant. In the panel data regression for 2006 to 2008, the transformed news story count becomes significant, while the independent variables for the main product becomes insignificant. Further, \cite{vlastakis} use company names as keyword in Google Trends and weekly news count. \cite{vlastakis} find that company name and weekly news count tends to be positively correlated, but do no further empirical comparison of the predictive power of the variables. 
\\\\
As mentioned earlier, most financial studies have only included one of the information variables in their analyses. \cite{joseph} use stock tickers, while \cite{neri} and \cite{bijl} use company names. \cite{tetlock} uses different transformed news counts to predict stock returns. \cite{moat} use number of Wikipedia edits and views of a company to investigate whether it can predict a company's stock returns.  
\\\\
Few studies have combined multiple measures of online activity, and we have found no empirical comparison of all the different measures. For Google Trends in particular, one must choose a keyword selection strategy. Several different strategies have been used with different theoretical arguments, but normally without an empirical comparison. In this paper we include all variables in our model to check whether each variable contributes the same information or improves the explanatory power of the model. We have collected several measures, and will be studying their relation to each other and stock market developments. Besides the previously mentioned indicators, we have also included the recently introduced Google Trends concept. 
\cleardoublepage

-hvilke variabler har blitt benyttet for å predikere vola, volum og ret
-argumenter for å benytte variablene(e)
	-wiki
    -news count
	-SVI (concept, ticker, search term)

vårt tilskudd:
-vi inkluderer alle variabler i modellen (får sjekket om det er samme informasjon til en modell eller forbedrer hver av variablene modellen ved å inkludere dem)
-Innenfor Google: 
	-vi inkluderer concept
	-kan sammenligne og se hvilken som er best og om de kan brukes
    samtidig, rettere sagt; om de inneholder ulik informasjon. 

 
