%===================================== CHAP 2 =================================

\chapter{Literature Review}
A generation ago, the efficient market hypothesis was widely accepted by economists. \cite{fama} emphasized the hypothesis that the capital markets are efficient, which means that the prices always fully reflect available information. Several studies in economic literature have investigated and concluded that this hypothesis is not true. \cite{malkiel} argues that the market cannot be perfectly efficient, as pricing irregularities and predictable patterns in stock returns can appear and persist for short time periods. \cite{grossman} argue that because information is costly, prices cannot perfectly reflect the available information, since if it did, investors would have not received compensation for the invested resources.
\\\\
Since information demand and supply is not directly available,  \cite{vlastakis} argue that studying how financial information affects the financial market requires proxies. Several theories for information markets exist, but until recently it has been hard to find good proxies for information flow. This has changed with the introduction of sources like Google Trends, online news databases, and Wikipedias Pageview Analysis. \cite{vlastakis} separate financial information flow into information demand and supply. Information supply tries to capture the amount of firm-related information being created or published. Information demand tries to capture the degree to which people are engaged in processing firm-related information.
\\\\
 Previous research has used several different measures for information demand. Among the most popular are Wikipedia views as used by \cite{moat} and different versions of Google Trends data. One of the most popular measures for information supply is daily or weekly news count, as used by \cite{vlastakis}.
\\\\
Several studies in economic literature have investigated the correlation between Google search volume and trading volume, volatility or stock return. Most of the economic literature find that search volume and trading volume is correlated, but cannot find any significant correlation between search volume and volatility or stock return. \cite{preis} find a significant correlation between transaction volume and Google search volume of the corresponding company names, but it seems that neither buying transactions nor selling transactions are preferred when one detects an increased search volume. \cite{engelberg} find that an increase in Google search volume index (SVI) predicts higher stock returns in the next two weeks and an eventual price reversal within the year. \cite{challet2014} do not find any clear evidence that Google Trends data contains more predictability than price returns themselves. \cite{vlastakis} find that variations in information demand is positively correlated to trading volume and volatility. \cite{neri} investigate whether Google Trends data can predict stock market activity in Norway, and find that Google Trends data neither predict nor correlate with stock returns.  
\\\\
To collect Google Trends data, a keyword (or concept) is needed for each company.  Several approaches to selecting keywords have been used. \cite{joseph} used each companies stock ticker. They argue that ticker trend is only used by someone who seriously is considering an investment, and that there are few other reasons for searching on company tickers. \cite{bijl} used search terms and argue that people searching for company information on the Internet are also those most likely to be interested in the company rather than an alternate meaning of the word.
\\\\
A few studies in economic literature have investigated the correlation between Wikipedia views or edits and trading volume, volatility and stock return. \cite{moat} find that the Wikipedia view based trading strategies have significantly higher returns than the random trading strategies, while the Wikipedia edit based strategies do not have significantly higher returns than the random trading strategies. 
\\\\
Some of the most popular measures for information supply is news count. \cite{preis2013} find a positive correlation between daily number of mentions of a company and the company's daily trading volume, both on the same day as the news is released and on the day before. \cite{ryan} find that company specific news have a significant impact on the corresponding company's price changes and trading volume. \cite{vlastakis} find a significant relationship between information supply, measured in number of firm-specific news per week, and realized volatility. 
\\\\
\cite{engelberg} find a correlation between Google search volume and news dummy variables. This result is consistent with the findings in \cite{vlastakis} that find a bidirectional causality and correlation between information demand and supply. 
\\\\
Few studies have combined multiple measures of online activity, and we have found no empirical comparison of them. For Google Trends in particular, one must choose a keyword selection strategy. Several different strategies have been used with different theoretical arguments, but normally without an empirical comparison. In this paper we have collected several measures, and will be studying their relation to each other and stock market developments. Besides the previously mentioned indicators, we have also included the recently introduced Google Trends concept. 
\cleardoublepage

 
