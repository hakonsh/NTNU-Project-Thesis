%===================================== CHAP 1 =================================

\chapter{Introduction}\label{sec:introduction}

A generation ago, the efficient market hypothesis was widely accepted by economists. \cite{fama} emphasized the hypothesis that the capital markets are efficient, which means that the prices fully reflect all available information. Several studies in economic literature have investigated and concluded that this hypothesis is not true. \cite{grossman} argue that because information is costly, prices cannot perfectly reflect all available information, since if it did, investors would  not receive compensation for resources invested in information gathering. Several theories tries to explain how supply and demand in the information market is related to the financial market. 
\todo{molnar:  Do not talk about efficient market hypothesis, try to motivate your paper more in lines of what this report is advising. RRHorizonMatters}
\\\\
\cite{vlastakis} separate financial information flow into information demand and supply. Information supply tries to capture the amount of firm-related information being created or published. Information demand tries to capture the degree to which people are engaged in processing firm-related information. Since information demand and supply is not directly available, \cite{vlastakis} argue that studying how financial information affects the financial market requires proxies. Until recently it has been hard to find good proxies for information flow. This has changed with the introduction of sources like Google Trends, online news databases, and Wikipedias Pageview Analysis.  
\\\\
Google Trends was introduced in 2008, and has since then been a popular proxy for information demand. Several studies in economic literature have investigated whether Google Trends search volumes can predict stock market development. Most of the economic literature find that search volume and trading volume is correlated, but cannot find any correlation with volatility or stock return. \cite{preis} find a significant correlation between transaction volume and Google search volume of the corresponding company names, but concludes that neither buying nor selling transactions are preferred when one detects an increased search volume. \cite{engelberg} find that an increase in Google search volume predicts higher stock returns in the next two weeks and an eventual price reversal within the year. \cite{challet2014} do not find any clear evidence that Google Trends data contain more explanatory power than price returns when predicting future financial returns. \cite{vlastakis} find that variations in Google search volume is positively correlated to trading volume and volatility. \cite{neri} investigate whether Google Trends data can predict stock market activity in Norway, and find that Google Trends data neither predict nor correlate with stock returns. 
\\\\
To collect Google Trends data, a keyword (or concept) is needed for each company. The two main keywords used in economic literature are stock tickers and company names, based on different arguments. \cite{joseph} use stock tickers. They argue that ticker trend is only used by someone interested in financial information, and that it therefore a better estimate of investor attention. \cite{bijl} and \cite{vlastakis} used search terms and argue that company names are less likely to have alternate meanings, which can be a problem with tickers (for example Caterpillar with the ticker "CAT"). \cite{engelberg} use tickers, because ticker trends only capture investor attention on trading and asset pricing, and that identifying a stock using its ticker also avoids the problem of alternate spellings of company names. In this article, we extend the literature by making an empirical comparison of the explanatory power of the different keywords for stock market movements. Google search probability is another measure for information demand as used by \todo{Hvorfor kommer dette i slutten av et avsnitt som handler om keyword selection?} \cite{vozlyublennaia}, who finds that attention can alter predictability of stock returns.       
\\\\
Another common measure for information demand is views and edits for a company's Wikipedia page. \cite{moat} find that the Wikipedia view based trading strategies have significantly higher returns than the random trading strategies, while the Wikipedia edit based strategies do not have significantly higher returns than the random trading strategies. 
\\\\
News article count is a popular measure of information supply. \cite{preis2013} find a positive correlation between daily number of mentions of a company and the company's daily trading volume, both on the same day as the news is released and on the day before. \cite{ryan} find that company specific news have a significant impact on the corresponding company's price changes and trading volume. \cite{vlastakis} find a significant relationship between the number of firm-specific news per week and realized volatility. \cite{tetlock} shows that public news predict lower ten-day reversals of daily stock returns and higher ten-day volume-induced momentum in daily returns.  
\\\\
Few studies have made an empirical comparison of the different information variables. Additionally, only a few studies have included more than one of the mentioned information variables in the analyses. \cite{vlastakis} and \cite{engelberg} are some of the few papers that have done this. \cite{engelberg} have collected three types of keywords in Google Trends: company name, ticker and searches on the main product of the company. They have also included different types of transformed news counts in their models. \cite{engelberg} find that stock ticker has little correlation with a news-based measure of investor sentiment. \cite{engelberg} perform panel data regressions to check the predictive power of Google Trends data and news count with stock return as dependent variable. They get inconsistent results between the different sampling periods they perform panel data regressions for. Further, \cite{vlastakis} use company names as keyword in Google Trends and weekly news count. \cite{vlastakis} find that company name and weekly news count tends to be positively correlated, but do no further empirical comparison of the predictive power of the two information variables. 
\\\\
In this paper we have collected both news counts, Wikipedia views, and several variations of Google trends data, including data from the recently introduced "concepts" function. We will be studying the variables relation to each other and stock market developments, to determine the information value they contribute individually and collectively.
\\\\

\todo{Add a paragraph with our findings here......}
