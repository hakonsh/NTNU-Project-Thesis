%===================================== CHAP 1 =================================

\chapter{Introduction}\label{sec:introduction}
Information market theory has existed for several years as information is extremely valuable in financial markets. Several papers have explored how different supply and demand indicators are related to the financial market. We will contribute to the literature by making an empirical comparison between several of the most common information supply and demand indicators. Before comparing the variables, it is interesting to have a look at the most important findings in the existing literature. 
\\\\
Financial information flow is usually separated into information demand and supply. These concepts are the fundamental basis in our thesis. Thus, it is important to clarify these concepts. Information supply tries to capture the amount of firm-related information being created or published. Information demand tries to capture the degree to which people are engaged in processing firm-related information. Since information demand and supply is not directly observable, \cite{vlastakis} argue that studying how financial information affects the financial market requires proxies. Until recently it has been hard to find good proxies for information flow. This has changed with the introduction of sources like Google Trends, online news databases, and Wikipedias Pageview Analysis.  
\\\\
Today, Google Trends is an important source of information demand. Google Trends was introduced in 2008, and has since then been a popular proxy for information demand. In this paper, we have collected Google Trends data for all the companies in our dataset. An important goal of this study is to investigate whether one Google Trends keyword strategy is preferable as predictor. But, before moving to this part it is interesting to have a look at whether Google Trends can predict stock market development at all. Several studies in economic literature have investigated this. Most of the economic literature find that search volume and trading volume is correlated, but cannot find any significant correlation with volatility or stock return. One of the researchers that find a significant correlation between transaction volume and Google search volume is \cite{preis}. However, the paper concludes that neither buying nor selling transactions are preferred when one detects an increased search volume. \cite{vlastakis} also find a significant correlation between Google search volume and trading volume, and between search volume and volatility. Further, stock returns turns out to be hard to predict by Google Trends data. However, \cite{engelberg} find that an increase in search volume predicts higher stock returns in the next two weeks and an eventual price reversal within the year. In contrast, \cite{challet2014} and \cite{neri} do not find any clear evidence that Google Trends data contain more explanatory power than price returns when predicting future financial returns. The conclusions indicate that the connection between Google Trends data and stock returns is rather weak, but that Google Trends data is a better predictor of trading volume. 
\\\\
As mentioned earlier, an important goal of the study, is to investigate what type of Google Trends keyword strategy, if any, that turns out to be the best predictor of financial data. To collect Google Trends data, a keyword is needed for each company. We separate Google Trends keyword strategies into three different types; concept trend, ticker trend and search term trend. Before we present our empirical comparison of the different Google Trends keyword strategies, it is interesting to investigate what the economic literature has found. The most frequently used Google Trends keyword strategies are ticker trend (stock ticker) and search term trend (company names). The papers that use ticker, among them \cite{joseph}, base this on the fact that ticker trend is more or less only used by someone interested in financial information, and that ticker trend therefore a better estimate of investor attention. \cite{engelberg} also use tickers, because ticker trends only capture investor attention on trading and asset pricing, and that identifying a stock using its ticker also avoids the problem of alternate spellings of company names. In contrast, \cite{bijl} and \cite{vlastakis} use search term trends and argue that company names are less likely to have alternate meanings, which can be a problem with tickers (for example Caterpillar with the ticker "CAT"). The newest Google Trend keyword strategy, concept trend, has been investigated to a low extent. In this article, we also investigate this keyword strategy in our empirical comparison of the explanatory power of the different keywords for stock market movements.     
\\\\
There are also other common measures for information other than Google Trends data. Among them, views and edits for a company's Wikipedia page. \cite{moat} investigate this information demand measure, and find that the Wikipedia view based trading strategies have significantly higher returns than the random trading strategies, while the Wikipedia edit based strategies do not have significantly higher returns than the random trading strategies. 
\\\\
So far we have only discussed the connection between information demand and stock market movements, but we will now turn our attention to information supply, and research on its predictive power with regards to financial data. One of the most popular measures of information supply is news article count. This is, among others, used by \cite{preis2013}, who find a positive correlation between daily number of mentions of a company and the company's daily trading volume, both on the same day as the news is released and on the day before. \cite{ryan} also investigate the connection between company specific news and financial data. They conclude that company specific news have a significant impact on the corresponding company's price changes and trading volume. \cite{vlastakis} is one of the few papers in the economic literature, that investigate the connection between both information demand and supply and financial data. Earlier, we mentioned that they found a correlation between information demand and stock market movements, but they also find a significant relationship between the number of firm-specific news per week and realized volatility. \cite{tetlock} shows that public news predict lower ten-day reversals of daily stock returns and higher ten-day volume-induced momentum in daily returns.  
\\\\
As mentioned, few studies have made an empirical comparison of the different information variables. Additionally, only a few studies have included more than one of the mentioned information variables in their analyses. \cite{vlastakis} and \cite{engelberg} are some of the few papers that have done this. \cite{engelberg} have collected three types of keywords in Google Trends: company name, ticker and searches on the main product of the company. They have also included different types of transformed news counts in their models. \cite{engelberg} find that stock ticker has little correlation with a news-based measure of investor sentiment. \cite{engelberg} perform panel data regressions to check the predictive power of Google Trends data and news count with stock return as dependent variable. They get inconsistent results between the different sampling periods they perform panel data regressions for. Further, \cite{vlastakis} use company names as keyword in Google Trends and weekly news count. They find that company name and weekly news count tends to be positively correlated, but do no further empirical comparison of the predictive power of the two information variables. 
\\\\
In this paper we have collected both news counts, Wikipedia views, and several variations of Google Trends data, including data from the recently introduced concept function. We will be studying the variables relation to each other and stock market developments, to determine the information value they contribute individually and collectively.
\\\\
Our results indicate that there is little to gain by including more than one information variable. When using only one variable, Google Trends outperforms Wikipedia views, and news article counts. Among the different Google Trend keywords, tickers performed the best, while the new Google Trends concept function outperforms company name. We also find evidence that Google Trends leads news counts, which strengthens the hypothesis that one good information demand variable should be enough. 